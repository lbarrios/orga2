% Ejercicio 3 - a)
\paragraph{replace-me-with-a-descriptive-text}\label{subsubsec:ej3-a}
Escribir una rutina que se encargue de limpiar el buffer de vídeo y pintarlo
como indica la figura 8. Tener en cuenta que deben ser escritos de forma
genérica para posteriormente ser completados con información del sistema. Además
considerar estas imágenes como sugerencias, ya que pueden ser modificadas a
gusto según cada grupo mostrando siempre la misma información.
\hruler
\fixme{Respuesta}

% Ejercicio 3 - b)
\paragraph{replace-me-with-a-descriptive-text}\label{subsubsec:ej3-b}
Escribir las rutinas encargadas de inicializar el directorio y tablas de páginas
para el kernel (mmu inicializar dir kernel). Se debe generar un directorio de
páginas que mapee, usando identity mapping, las direcciones 0x00000000 a
0x00DC3FFF, como ilustra la figura 5. Además, esta función debe inicializar el
directorio de páginas en la dirección 0x27000 y las tablas de páginas según
muestra la figura 1.
\hruler
\fixme{Respuesta}

% Ejercicio 3 - c)
\paragraph{replace-me-with-a-descriptive-text}\label{subsubsec:ej3-c}
Completar el código necesario para activar paginación.
\hruler
\fixme{Respuesta}

% Ejercicio 3 - d)
\paragraph{replace-me-with-a-descriptive-text}\label{subsubsec:ej3-d}
Escribir una rutina que imprima el nombre del grupo en pantalla. Debe estar
ubicado en la primer linea de la pantalla alineado a derecha.
\hruler
\fixme{Respuesta}
