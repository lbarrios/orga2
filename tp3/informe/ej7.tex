% Ejercicio 7 - a)
\paragraph{replace-me-with-a-descriptive-text}\label{subsubsec:ej7-a}
Construir una función para inicializar las estructuras de datos del scheduler.
\hruler
\fixme{Respuesta}

% Ejercicio 7 - b)
\paragraph{replace-me-with-a-descriptive-text}\label{subsubsec:ej7-b}
Crear la función sched proximo indice() que devuelve el índice en la GDT de la
próxima tarea a ser ejecutada. Construir la rutina de forma devuelva el indice
de la TSS1 y luego el de la TSS2 de forma intercalada, para dos tareas fijas.
\hruler
\fixme{Respuesta}

% Ejercicio 7 - c)
\paragraph{replace-me-with-a-descriptive-text}\label{subsubsec:ej7-c}
Modificar la rutina de la interrupción 0x52, para que implemente los tres
servicios del sistema según se indica en la sección 3.1.1.
\hruler
\fixme{Respuesta}

% Ejercicio 7 - d)
\paragraph{replace-me-with-a-descriptive-text}\label{subsubsec:ej7-d}
Modificar el código necesario para que se realice el intercambio de tareas por
cada ciclo de reloj. El intercambio se realizará según indique la función sched
proximo indice().
\hruler
\fixme{Respuesta}

% Ejercicio 7 - e)
\paragraph{replace-me-with-a-descriptive-text}\label{subsubsec:ej7-e}
Modificar la función sched proximo indice() de forma que ejecute todas las
tareas según se describe en la sección 3.2.
\hruler
\fixme{Respuesta}

% Ejercicio 7 - f)
\paragraph{replace-me-with-a-descriptive-text}\label{subsubsec:ej7-f}
Modificar las rutinas de excepciones del procesador para que impriman el
problema que se produjo en pantalla, desalojen a la tarea que estaba corriendo y
corran la próxima, indicando en pantalla porque razón fue desalojada la tarea en
cuestión. Nota: Se recomienda construir funciones en C que ayuden a resolver
problemas como convertir direcciones de el mapa a direcciones físicas.
\hruler
\fixme{Respuesta}
