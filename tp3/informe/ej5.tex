% Ejercicio 5 - a)
\paragraph{replace-me-with-a-descriptive-text}\label{subsubsec:ej5-a}
Completar las entradas necesarias en la IDT para asociar una rutina a la
interrupción del reloj, otra a la interrupción de teclado y por último una a la
interrupción de software 0x52.
\hruler
\fixme{Respuesta}

% Ejercicio 5 - b)
\paragraph{replace-me-with-a-descriptive-text}\label{subsubsec:ej5-b}
Escribir la rutina asociada a la interrupción del reloj, para que por cada tick
llame a la función screen proximo reloj. La misma se encarga de mostrar cada vez
que se llame, la animación de un cursor rotando en la esquina inferior derecha
de la pantalla. La función proximo reloj está definida en isr.asm.
\hruler
\fixme{Respuesta}

% Ejercicio 5 - c)
\paragraph{replace-me-with-a-descriptive-text}\label{subsubsec:ej5-c}
Escribir la rutina asociada a la interrupción de teclado de forma que si se
presiona cualquier número, se presente el mismo en la esquina superior derecha
de la pantalla. El número debe ser escrito en color blanco con fondo de color
aleatorio por cada tecla que sea presionada\footnote{http://wiki.osdev.org/Text
UI}.
\hruler
\fixme{Respuesta}

% Ejercicio 5 - d)
\paragraph{replace-me-with-a-descriptive-text}\label{subsubsec:ej5-d}
Escribir la rutina asociada a la interrupción 0x52 para que modifique el valor
de eax por 0x42. Posteriormente este comportamiento va a ser modificado para
atender los servicios del sistema.
\hruler
\fixme{Respuesta}
