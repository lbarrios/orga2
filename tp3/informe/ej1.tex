% Ejercicio 1 - a)
\paragraph{GDT Básica}\label{subsubsec:ej1-a}
Completar la Tabla de Descriptores Globales (GDT) con 4 segmentos, dos para
código de nivel 0 y 3; y otros dos para datos de nivel 0 y 3. Estos segmentos
deben direccionar los primeros 733MB de memoria. En la gdt, por restricción del
trabajo práctico, las primeras 8 posiciones se consideran utilizadas y no
deben utilizarse. El primer índice que deben usar para declarar los segmentos,
es el 9 (contando desde cero).
\hruler

La GDT - $Global$ $Descriptors$ $Table$ - es una tabla de descriptores de segmento, en donde cada entrada describe
un segmento particular, y tiene la siguiente forma:

\begin{figure}[H]
\begin{center}
\includegraphics[width=9cm]{imagenes/gdt.png}
\end{center}
\end{figure}

En éste ejercicio vamos a definir 4 segmentos, cada uno con su entrada en la GDT.

\subparagraph*{Entrada 1}
\begin{itemize}
	
    \item $.base_{0-15}$ = 0,

    \item $.base_{23-16}$ = 0,
    \item $.base_{31-24}$ = 0,
    
El segmento empieza en la dirección de memoria 0.
        
    \item $.limit_{0-15}$ = 0xDCFF,
    \item $.limit_{16-19}$ = 0x2;

733 MB = 187648 bloques de 4 KB.
\\ El máximo offset que queremos utilizar es 187648 - 1 = 0x2DCFF 
    
    \item .type = 8
        
Será un segmento de código $Execute-Only$.
        
    \item .s  = 1
        
Sera un segmento de Código/Datos.
        
    \item .dpl = 0
        
Este segmento tendrá nivel de privilegio 0.

    \item .p = 1
        
Este segmento estará presente en memoria.
    
    \item .avl = 0
        
A disposición del programador.


    \item .l = 0
        
Este campo, que figura como 0 en el diagrama, sólo va en 1 cuando se trabaja en modo IA-32e.

    \item .db = 1
        
Siginifica que el tamaño de las operaciones va a ser de 32 bits.

    \item .g = 1

Nuestro límite lo vamos a expresar con una granularidad de 4KB en vez de 1B. Es decir, vamos a poder direccionar (limite - 1) * 4K bytes.

\end{itemize}

\subparagraph*{Entrada 2}

Se diferencia de la entrada 1 en que el campo $dpl$ - el nivel de privilegio - se pone en 3.

\subparagraph*{Entrada 3}

Se diferencia de la entrada 1 en que el campo $type$ se pone en 2 - $Read-Write$.

\subparagraph*{Entrada 4}

Se diferencia de la entrada 3 en que el campo $dpl$ se pone en 3.




% Ejercicio 1 - b)
\paragraph{Modo Protegido}\label{subsubsec:ej1-b}
Completar el código necesario para pasar a modo protegido y setear la pila del
kernel en la dirección 0x27000.
\hruler

Primero cargamos la GDT, pasandole su descriptor como parámetro a la instrucción LGDT. Deshabilitamos las interrupciones y seteamos el bit PE del
registro $CR0$ en 1, pasando finalmente a modo protegido.
Habilitamos el el pin A20 para poder acceder a direcciones de memoria superiores al MB.
En éste momento se hace un $FAR$ $JUMP$ hacia la siguiente instrucción, utilizando como selector de segmento el offset en la GDT de la primer 
entrada definida más arriba (de código de nivel de administrador).
Se procede a cargar los selectores de segmento y setear la base de la pila.

% Ejercicio 1 - c)
\paragraph{GDT - Área de Memoria}\label{subsubsec:ej1-c}
Declarar un segmento adicional que describa el área de la pantalla en memoria
que pueda ser utilizado sólo por el kernel.
\hruler

En la GDT ya habíamos definido una quinta entrada para el segmento de la memoria de video.

Tiene la base en la posición 0xB8000. Como sólo va a direccionar \fixme{}

% Ejercicio 1 - d)
\paragraph{Limpiar la pantalla}\label{subsubsec:ej1-d}
Escribir una rutina que se encargue de limpiar la pantalla y pintar el area de
el mapa un fondo de color (sugerido verde). Para este ejercicio se debe escribir
en la pantalla usando el segmento declarado en el punto anterior (para los
próximos ejercicios se accederá a la memoria de vídeo por medio del segmento de
datos de 773MB).

Nota: La GDT es un arreglo de gdt entry declarado sólo una vez como gdt. El
descriptor de la GDT en el código se llama GDT DESC.

\hruler
La pantalla en este caso es una matriz en donde cada posición es un caracter ASCII con un determinado color de letra y fondo. La rutina es un
simple ciclo que va escribiendo el caracter nulo con un fondo verde sobre las posiciones de (0, 0) a (50, 50), que representan el campo de
batalla.
