% Ejercicio 4 - a)
\paragraph{replace-me-with-a-descriptive-text}\label{subsubsec:ej4-a}
Escribir una rutina (inicializar mmu) que se encargue de inicializar las
estructuras necesarias para administrar la memoria en el area libre.
\hruler
\fixme{Respuesta}

% Ejercicio 4 - b)
\paragraph{replace-me-with-a-descriptive-text}\label{subsubsec:ej4-b}
Escribir una rutina (mmu inicializar dir tarea) encargada de inicializar un
directorio de páginas y tablas de páginas para una tarea, respetando la figura
5. La rutina debe copiar el código de la tarea a su área asignada, es decir sus
dos páginas de código dentro de el mapa y mapear dichas páginas a partir de la
dirección virtual 0x08000000(128MB).
\hruler
\fixme{Respuesta}

% Ejercicio 4 - c)
\paragraph{replace-me-with-a-descriptive-text}\label{subsubsec:ej4-c}
Escribir dos rutinas encargadas de mapear y desmapear páginas de memoria.

I- mmu mapear pagina(unsigned int virtual, unsigned int cr3, unsigned int
fisica) Permite mapear la página física correspondiente a fisica en la dirección
virtual virtual utilizando cr3.

II- mmu unmapear pagina(unsigned int virtual, unsigned int cr3). Borra el mapeo
creado en la dirección virtual virtual utilizando cr3.
\hruler
\fixme{Respuesta}

% Ejercicio 4 - d)
\paragraph{replace-me-with-a-descriptive-text}\label{subsubsec:ej4-d}
Construir un mapa de memoria para tareas e intercambiarlo con el del kernel,
luego cambiar el color del fondo del primer caracter de la pantalla y volver a
la normalidad.

Nota: Por la construcción del kernel, las direcciones de los los mapas de
memoria (page directory y page table) están mapeadas con identity mapping. En
los ejercicios en donde se modifica el directorio o tabla de páginas, hay que
llamar a la función tlbflush para que se invalide la cache de traducción de
direcciones.
\hruler
\fixme{Respuesta}
