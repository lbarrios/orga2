% Ejercicio 8 - a)
\paragraph{replace-me-with-a-descriptive-text}\label{subsubsec:ej8-a}
Crear un tanque (tarea) propio que mapee paginas a muerte contra otros
intrepidos tanques. Para esto pueden editar el código del primer tanque a gusto.
El tanque debe tener las siguientes características,

\begin{itemize}
  \item No ocupar más de 8 kb (tener en cuenta la pila).
  \item Tener como punto de entrada la dirección cero.
  \item Estar compilada para correr desde la dirección 0x08000000.
  \item Utilizar solo los servicios presentados en el trabajo práctico.
\end{itemize}

Explicar en pocas palabras qué estrategia utilizaron en su tanque en términos de
<<defensa>> y <<ataque>>.
\hruler
\fixme{Respuesta}

% Ejercicio 8 - b)
\paragraph{replace-me-with-a-descriptive-text}\label{subsubsec:ej8-b}
Si consideran que su tanque es capaz de enfrentarse contra los tanques del
resto de sus compañeros, pueden enviar el binario a la lista de docentes
indicando los siguientes datos,

\begin{itemize}
  \item Nombre del tanque, ej: \emph{T-90S Bhishma}
  \item Características letales, ej: \emph{Cañón 2A46M de 125 mm con cargador 
    automático}
  \item Sistema de defensa, ej: \emph{1.350 mm en blindaje laminado, blindaje 
    reactivo...}
\end{itemize}

Se realizará una competencia a fin de cuatrimestre con premios en/de chocolate
para los primeros puestos.
\hruler
\fixme{Respuesta}

% Ejercicio 8 - c)
\paragraph{replace-me-with-a-descriptive-text}\label{subsubsec:ej8-c}
Mencionar la mayor cantidad de características del tanque montado en el jardín
del edificio Libertador en la ciudad de Buenos Aires.
\hruler
\fixme{Respuesta}
