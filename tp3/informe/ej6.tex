% Ejercicio 6 - a)
\paragraph{replace-me-with-a-descriptive-text}\label{subsubsec:ej6-a}
Definir 3 entradas en la GDT para ser usadas como descriptores de TSS. Una será
reservada para la tarea inicial y otras dos para realizar el intercambio entre
tareas, denominadas TSS1 y TSS2 respectivamente.
\hruler
\fixme{Respuesta}

% Ejercicio 6 - b)
\paragraph{replace-me-with-a-descriptive-text}\label{subsubsec:ej6-b}
Completar la entrada de la TSS1 con la información de la tarea Idle. Esta
información se encuentra en el archivo TSS.C. La tarea Idle se encuentra en la
dirección 0x00020000. La pila se alojará en la misma dirección que la pila del
kernel y será mapeada con identity mapping. Esta tarea ocupa 2 paginas de 4KB y
debe ser mapeada con identity mapping. Además la misma debe compartir el mismo
CR3 que el kernel.
\hruler
\fixme{Respuesta}

% Ejercicio 6 - c)
\paragraph{replace-me-with-a-descriptive-text}\label{subsubsec:ej6-c}
Completar el resto de la información correspondiente a cada tarea en la
estructura auxiliar de contextos. El código de las tareas se encuentra a partir
de la dirección 0x00010000 ocupando dos páginas de 4kb cada una. El mismo debe
ser mapeado a partir de la dirección 0x08000000. Para la dirección de la pila se
debe utilizar el mismo espacio de la tarea, la misma crecerá desde la base de la
tarea. Para el mapa de memoria se debe construir uno nuevo para cada tarea
utilizando la función mmu inicializar dir usuario. Además, tener en cuenta que
cada tarea utilizará una pila distinta de nivel 0, para esto se debe pedir una
nueva pagina libre a tal fin.
\hruler
\fixme{Respuesta}

% Ejercicio 6 - d)
\paragraph{replace-me-with-a-descriptive-text}\label{subsubsec:ej6-d}
Completar la entrada de la GDT correspondiente a la tarea inicial.
\hruler
\fixme{Respuesta}

% Ejercicio 6 - e)
\paragraph{replace-me-with-a-descriptive-text}\label{subsubsec:ej6-e}
Completar la entrada de la GDT correspondiente a la TSS1, que contiene la
información de la tarea Idle.
\hruler
\fixme{Respuesta}

% Ejercicio 6 - f)
\paragraph{replace-me-with-a-descriptive-text}\label{subsubsec:ej6-f}
Completar la entrada de la GDT correspondiente a la TSS2.
\hruler
\fixme{Respuesta}

% Ejercicio 6 - g)
\paragraph{replace-me-with-a-descriptive-text}\label{subsubsec:ej6-g}
Escribir el código necesario para ejecutar la tarea Idle, es decir, saltar
intercambiando las TSS, entre la tarea inicial y la tarea Idle.

Nota: En tss.c están definidas las tss como estructuras TSS. Trabajar en tss.c y
kernel.asm.
\hruler
\fixme{Respuesta}
